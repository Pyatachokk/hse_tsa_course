\documentclass[a4paper,12pt]{article}

%%% Работа с русским языком
\usepackage{cmap}					% поиск в PDF
\usepackage{mathtext} 				% русские буквы в формулах
\usepackage[T2A]{fontenc}			% кодировка
\usepackage[utf8]{inputenc}			% кодировка исходного текста
\usepackage[english,russian]{babel}	% локализация и переносы

%%% Дополнительная работа с математикой
\usepackage{amsfonts,amssymb,amsthm,mathtools} % AMS
\usepackage{amsmath}
\usepackage{icomma} % "Умная" запятая: $0,2$ --- число, $0, 2$ --- перечисление

\usepackage[left = 2cm, right = 2cm, top = 2cm, bottom = 2cm]{geometry}

\usepackage{graphicx}

%% Номера формул
%\mathtoolsset{showonlyrefs=true} % Показывать номера только у тех формул, на которые есть \eqref{} в тексте.

%% Шрифты
\usepackage{euscript}	 % Шрифт Евклид
\usepackage{mathrsfs} % Красивый матшрифт

%% Свои команды
\DeclareMathOperator{\sgn}{\mathop{sgn}}

%% Перенос знаков в формулах (по Львовскому)
\newcommand*{\hm}[1]{#1\nobreak\discretionary{}
	{\hbox{$\mathsurround=0pt #1$}}{}}

%%% Работа с картинками
\usepackage{graphicx}  % Для вставки рисунков
\graphicspath{{images/}{images2/}}  % папки с картинками
\setlength\fboxsep{3pt} % Отступ рамки \fbox{} от рисунка
\setlength\fboxrule{1pt} % Толщина линий рамки \fbox{}
\usepackage{wrapfig} % Обтекание рисунков и таблиц текстом

%%% Работа с таблицами
\usepackage{array,tabularx,tabulary,booktabs} % Дополнительная работа с таблицами
\usepackage{longtable}  % Длинные таблицы
\usepackage{multirow} % Слияние строк в таблице
\usepackage{upgreek}
\usepackage{enumerate}
\usepackage{ dsfont }
\usepackage[weather]{ifsym}

%%% Цветной текст

\usepackage{movie15}
% in preamble
\usepackage{graphicx}
\usepackage{animate}

\usepackage[usenames]{color}
\usepackage{colortbl}

%%% Гиперссылки

\usepackage{xcolor}
\usepackage{hyperref}
\definecolor{linkcolor}{HTML}{199B03} % цвет ссылок
\definecolor{urlcolor}{HTML}{199B03} % цвет гиперссылок

\hypersetup{pdfstartview=FitH,  linkcolor=linkcolor,urlcolor=urlcolor, colorlinks=true}


%%% Заголовок
\author{Zekhov Matvei}
\title{Title}
\date{\today}


\begin{document}

	Работу следует представить в виде отчёта в pdf или html формате. B начале работы должен идти текст с графиками, в конце работы в качестве приложения должен идти код. Общий объем текста (без приложений) должен составлять не более 10 страниц.
	
	Вставлять скрины рукописных формул в текст категорически запрещается. Если очень хочется ускорить свою работу, можете попробовать воспользоваться Mathpix, он умеет конвертировать рукописные формулы в тех. Но по опыту руками быстрее и полезнее. За выдающееся оформление в техе может быть добавлено до 1 балла из 10 на усмотрение проверяющего.
	
	В отличие от домашнего задания формулировка проекта во многих пунктах является вольной и предполагает творческий подход. В каждом пункте можно сделать больше, чем указанный минимум.
	Проект можно выполнять в одиночку, а также группой из двух или трёх человек.
	Исходные данные должны быть прикреплены к работе или загружаться автоматически из открытых источников в интернете.
	
	
	\begin{enumerate}
	

	\item Возьмите любой ряд с ежемесячными наблюдениями.
	\item (30 баллов) Визуализируйте сам ряд, ряд обычных и сезонных разностей, компоненты ряда, обычные и частные автокорреляционные функции.
	Прокомментируйте графики.
	\item (10 баллов) Является ряд стационарным?
	Подтвердите ответ подходящим тестом.
	\item (10 баллов) Если разумно применить к исходному ряду какое-либо преобразование, то примените его, мотивировав свой выбор.
	\item (5 баллов) Поделите ряд на тестовую и обучающую выборку.
	\item (60 баллов) Оцените ряд моделей/алгоритмов на тестовой выборке.
	Здесь вы ограничены только вашей фантазией! Как минимум следует рассмотреть: наивную модель, ETS, SARIMA, случайный лес, тета-метод и усреднение моделей-лидеров. Можно в качестве предикторов взять дополнительные ряды. Можно прогнозировать компоненты ряда разными моделями.
	\item (10 баллов) Выберите наилучшую модель.
	Визуализируйте ряд остатков наилучшей модели на обучающей выборке, прокомментируйте. Постройте график прогнозов на два года вперед, переоценив наилучшую моделей по полной выборке. Не забудьте вернуться к исходному ряду, если вы делали преобразование.
	\item (20 баллов) Удивите проверяющих реализацией какой-нибудь интересной дополнительной идеи.
	Можно заполнить пропуски, проверить наличие структурных сдвигов, выявить аномальные наблюдения, можно использовать техники не упомянутые в курсе.
	\item (бонус, 10 баллов) Если вы занимались прогнозированием рядов вне данного курса, кратко опишите проблемы с которыми вам пришлось столкнуться и как вы их решали. Если не занимались, то расскажите, какие сюжеты курса дались легко, а какие вызывают сложности.
	\end{enumerate}
\end{document}
