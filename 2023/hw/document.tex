\documentclass[a4paper,12pt]{article}

%%% Работа с русским языком
\usepackage{cmap}					% поиск в PDF
\usepackage{mathtext} 				% русские буквы в формулах
\usepackage[T2A]{fontenc}			% кодировка
\usepackage[utf8]{inputenc}			% кодировка исходного текста
\usepackage[english,russian]{babel}	% локализация и переносы

%%% Дополнительная работа с математикой
\usepackage{amsfonts,amssymb,amsthm,mathtools} % AMS
\usepackage{amsmath}
\usepackage{icomma} % "Умная" запятая: $0,2$ --- число, $0, 2$ --- перечисление

\usepackage[left = 2cm, right = 2cm, top = 2cm, bottom = 2cm]{geometry}

\usepackage{graphicx}

%% Номера формул
%\mathtoolsset{showonlyrefs=true} % Показывать номера только у тех формул, на которые есть \eqref{} в тексте.

%% Шрифты
\usepackage{euscript}	 % Шрифт Евклид
\usepackage{mathrsfs} % Красивый матшрифт

%% Свои команды
\DeclareMathOperator{\sgn}{\mathop{sgn}}

%% Перенос знаков в формулах (по Львовскому)
\newcommand*{\hm}[1]{#1\nobreak\discretionary{}
	{\hbox{$\mathsurround=0pt #1$}}{}}

%%% Работа с картинками
\usepackage{graphicx}  % Для вставки рисунков
\graphicspath{{images/}{images2/}}  % папки с картинками
\setlength\fboxsep{3pt} % Отступ рамки \fbox{} от рисунка
\setlength\fboxrule{1pt} % Толщина линий рамки \fbox{}
\usepackage{wrapfig} % Обтекание рисунков и таблиц текстом

%%% Работа с таблицами
\usepackage{array,tabularx,tabulary,booktabs} % Дополнительная работа с таблицами
\usepackage{longtable}  % Длинные таблицы
\usepackage{multirow} % Слияние строк в таблице
\usepackage{upgreek}
\usepackage{enumerate}
\usepackage{ dsfont }
\usepackage[weather]{ifsym}

%%% Цветной текст

\usepackage{movie15}
% in preamble
\usepackage{graphicx}
\usepackage{animate}

\usepackage[usenames]{color}
\usepackage{colortbl}

%%% Гиперссылки

\usepackage{xcolor}
\usepackage{hyperref}
\definecolor{linkcolor}{HTML}{199B03} % цвет ссылок
\definecolor{urlcolor}{HTML}{199B03} % цвет гиперссылок

\hypersetup{pdfstartview=FitH,  linkcolor=linkcolor,urlcolor=urlcolor, colorlinks=true}


%%% Заголовок
\author{Zekhov Matvei}
\title{Title}
\date{\today}


\begin{document}
		Работу следует представить в виде отчёта в pdf формате. В начале работы должен идти текст с графиками, в конце работы в качестве приложения должен идти код. За графики без подписанных осей и заголовков оценка будет снижена. Общий объем текста (без приложений) должен составлять не более 10 страниц.
	\begin{enumerate}
			\item (10 баллов) Рассмотрим $M A(2)$ процесс $y_t=10+u_t+3 u_{t-1}$, где величины $u_t$ независимы и нормально распределены $\mathcal{N}(0 ; 4)$.
		\begin{enumerate}
				\item Рассчитайте теоретическую автокорреляционную функцию процесса $A C F, \rho_k$.
			\item Рассчитайте первые два значения частной автокорреляционной функции $P A C F, \phi_{11}, \phi_{22}$.
			\item Сгенерируйте траекторию данного процесса длиной 30 наблюдений. Постройте график ряда, график первых десяти значений выборочной $A C F$ и $P A C F$.
			\item Повторите предыдущий пункт для 300 наблюдений. Верно ли, что с ростом числа наблюдений выборочная $A C F$ сходится к истинной $A C F$, а выборочная $P A C F$ к истинной $P A C F$ ?
		\end{enumerate}
	
	\item (10 баллов) Рассмотрим случайное блуждание $y_t=1+y_{t-1}+u_t$, где величины $u_t$ независимы и нормально распределены $\mathcal{N}(0 ; 4)$, а $y_0=10$.
	
	\begin{enumerate}
		\item Рассчитайте $\mathbb{E}\left(y_t\right), \mathbb{V} \operatorname{ar}\left(y_t\right), \operatorname{Cov}\left(y_{10}, y_{20}\right)$.
		\item Сравните $\operatorname{Corr}\left(y_{10}, y_{20}\right)$ и $\operatorname{Corr}\left(y_{110}, y_{120}\right)$.
		\item Сгенерируйте траекторию данного процесса длиной 30 наблюдений. Постройте график ряда, график первых десяти значений выборочной $A C F$ и $P A C F$.
		\item Повторите предыдущий пункт для 300 наблюдений. Верно ли, что с ростом числа наблюдений выборочная $A C F$ сходится к истинной $A C F$, а выборочная $P A C F$ к истинной $P A C F$ ?
	\end{enumerate}

	\item (20 баллов) Возьмите любой несезонный ряд годовой периодичности. Можно взять ряд с httрs: //fedstat.ru/, http://sophist.hse.ru/ или других источников.
	
	\begin{enumerate}
		\item Постройте график ряда, графики выборочных $A C F$ и $P A C F$.
		\item Визуально оцените, есть ли тренд? Похож ли процесс на стационарный?
		\item Оцените для ряда $\operatorname{ETS}(A A N)$ модель.
		\item Выпишите полученные уравнение, использовав оценённые значения параметров вместо параметров.
		\item Получите $80 \%$-й доверительный интервал на один и два шага вперёд «руками», исходя из выписанных уравнений.
		\item Получите 80\%-й доверительный интервал на один и два шага вперёд встроенными функциями.
		\item Постройте график прогноза и сам ряд.
	\end{enumerate}
	\newpage
	\item (30 баллов) Возьмите любой сезонный ряд квартальной или месячной периодичности.
	
	\begin{enumerate}
	\item Постройте разложение ряда на составляющие, используя $S T L$ алгоритм. Визуализируйте результат для трех разных значений силы сглаживания сезонности. Кратко прокомментируйте.
	\item Постройте разложение ряда на составляющие, используя $\operatorname{ETS}(A A A)$ модель.
	\item Разделите данные на обучающую и тестовую выборку, выделив на тестовую выборку два года наблюдений.
	\item Оцените $E T S(A A A), E T S(M A M)$, сезонную наивную модель и примените тета-метод с $S T L$ разложением по умолчанию и $\operatorname{ETS} S(A A A)$ для логарифма ряда.
	\item Для каждого подхода найдите $M A S E$ на тестовой выборке.
	\item Постройте прогноз, усредняющий прогнозы двух лидирующих по $M A S E$ подхода. Удалось ли обыграть два усредняемых подхода?
	
	\end{enumerate}
	

	
	\end{enumerate}

	

\end{document}
